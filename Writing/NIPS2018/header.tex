\usepackage{times}
\usepackage{titlesec}
\usepackage{longtable}
\usepackage{tikz}
\usepackage{wrapfig}
\usepackage{lipsum} 
\usepackage{hhline}
\usepackage{color}
%\usepackage[table]{xcolor}

\usepackage[amssymb]{SIunits}
\usepackage{latexsym}
\usepackage{multirow}
\usepackage{wrapfig}
\usepackage{comment}
\usepackage{import}
\usepackage[ruled, vlined, linesnumbered]{algorithm2e}
\usepackage{amsmath, amsthm, amssymb}
\usepackage{amsfonts}
\usepackage[format=plain,indention=0cm, font=small, labelfont=bf]{caption}
\usepackage{fancyhdr}
%\usepackage[pdftex]{graphicx}
\usepackage[pdftex,
	    colorlinks, 	
	    pdfstartview=FitH,
	    linkcolor=black,
	    citecolor=black,
	    urlcolor=black,
	    filecolor=black
	    ]{hyperref}
\usepackage{lscape}
\usepackage[T1]{fontenc}
\usepackage{floatrow}
\usepackage{enumerate}
\usepackage{pifont}
\usepackage{soul}
\usepackage{esvect}
\usepackage{enumitem}

%\usepackage[dvipsnames]{xcolor}


\usepackage{floatflt}
\usepackage{tabularx}
\usepackage{ragged2e}
\newcolumntype{Y}{ >{\RaggedRight\arraybackslash}X}
\newcommand\T{\rule{0pt}{2.6ex}}      
\newcommand\B{\rule[-1.2ex]{0pt}{0pt}}
\RequirePackage{color}
\definecolor{RED}{rgb}{1,0,0}
\definecolor{BLUE}{rgb}{0,0,1}
\definecolor{White}{rgb}{1,1,1}
\newtheorem{theorem}{Theorem}

%\newcommand{\idx}{CECI}
\newtheorem{thm}{Theorem}
\newtheorem{lem}{Lemma}
%\newtheorem{rule}{Rule}
%\newtheorem{lem}{Lemma}





\newcommand{\sub}{CECI}
\newcommand{\idx}{CECI}
%\newtheorem{example}{Example}
%\newtheorem{lem}{Example}
\newcommand{\simd}{SIMD-X}

\theoremstyle{expln}
\newtheoremstyle{expln}% <name>
{5pt}% <Space above>
{5pt}% <Space below>
{}% <Body font>
{}% <Indent amount>
{\itshape}% <Theorem head font>
{:}% <Punctuation after theorem head>
{}% <Space after theorem headi>
{}% <Theorem head spec (can be left empty, meaning `normal')>
% DOI
%\acmDOI{10.475/123_4}
\newtheorem{expln}{Explain}


%\theoremstyle{rul}
\newtheoremstyle{mystyle}% <name>
{.05in}% <Space above>
{0in}% <Space below>
{\itshape}% <Body font>
{}% <Indent amount>
{\bfseries}% <Theorem head font>
{: }% <Punctuation after theorem head>
{0in}% <Space after theorem headi>
{}% <Theorem head spec (can be left empty, meaning `normal')>
% DOI
%\acmDOI{10.475/123_4}
\theoremstyle{mystyle}
\newtheorem{rul}{Rule}

\newtheoremstyle{taskstyle}% <name>
{.05in}% <Space above>
{0in}% <Space below>
{}% <Body font>
{}% <Indent amount>
{\bfseries}% <Theorem head font>
{.}% <Punctuation after theorem head>
{.1in}% <Space after theorem headi>
{}% <Theorem head spec (can be left empty, meaning `normal')>
% DOI
%\acmDOI{10.475/123_4}
\theoremstyle{taskstyle}
\newtheorem{tstyle}{Research Task}


\definecolor{unitednationsblue}{rgb}{0.36, 0.57, 0.9}
\definecolor{pinkorange}{rgb}{1.0, 0.6, 0.4}
\definecolor{darkseagreen}{rgb}{0.56, 0.74, 0.56}


%\newtheoremstyle{examp}% <name>
%{5pt}% <Space above>
%{5pt}% <Space below>
%{\itshape}% <Body font>
%{}% <Indent amount>
%{\itshape}% <Theorem head font>
%{:}% <Punctuation after theorem head>
%{.5em}% <Space after theorem headi>
%{}% <Theorem head spec (can be left empty, meaning `normal')>
% DOI
%\acmDOI{10.475/123_4}
%\theoremstyle{examp}
%\newtheorem{examp}{Example}
%\newcommand{\td}{{}}
\newcommand{\td}{{\bf\color{red} FIXME~}}
\providecommand{\TODO}[1]{{\protect\color{red}\noindent {\bf [TODO]}\emph{#1} {\bf [/TODO]}}}
\providecommand{\todo}[1]{{\protect\color{red}\noindent {\bf [TODO]}\emph{#1} {\bf [/TODO]}}}
\providecommand{\add}[1]{{\protect\color{red} #1 (ADD) }}
\providecommand{\ADD}[1]{{\protect\color{red} #1 (ADD) }}
\providecommand{\CHECK}[1]{{\protect\color{red} #1 (check) }}
\providecommand{\check}[1]{{\protect\color{red} #1 (check) }}
\providecommand{\DummyText}[1]{{\protect\color{white} #1}}
%\renewcommand{\refname}{References Cited}
\newcommand{\degrees}{$\!\!$\char23$\!$}
\def\rrr#1\\{\par
\medskip\hbox{\vbox{\parindent=2em\hsize=6.12in
\hangindent=4em\hangafter=1#1}}}
\def\baselinestretch{1}

\newcommand\crule[3][black]{\textcolor{#1}{\rule{#2}{#3}}}

%\titlespacing\section{0pt}{1pt plus 0pt minus 2pt}{0pt plus 0pt minus 2pt}
\titlespacing\subsection{0pt}{8pt plus 4pt minus 2pt}{8pt plus 4pt minus 2pt}
%\titlespacing\subsubsection{0pt}{1pt plus 0pt minus 2pt}{0pt plus 0pt minus 2pt}

\newcommand\encircle[1]{%
	\tikz[baseline=(X.base)] 
	\node (X) [draw, shape=circle, inner sep=0, fill=black, text=white] {\strut #1};}